\usepackage{MonashWhite}

\usepackage{amsmath,bm,booktabs,tikz}
\usepackage{animate}
\usepackage{tikz}

\setbeamercolor{description item}{fg=Orange}

\def\pred#1#2#3{\hat{#1}_{#2|#3}}
\def\damped{$_\text{d}$}
\def\h+{h_{m}^{+}}
\def\st#1{\rlap{#1}\textcolor{red}{\rule{1cm}{0.1cm}}}

\graphicspath{{figs/}}

% Monash title page
\setbeamerfont{title}{series=\bfseries,parent=structure,size={\fontsize{26}{30}}}
\setbeamertemplate{title page}
{\placefig{-0.01}{-0.01}{width=1.01\paperwidth,height=1.01\paperheight}{figs/FHSWF-TitleSlide}
\begin{textblock}{7.5}(1,2)\fontsize{20}{30}\sf
{\color{white}\raggedright\usebeamerfont{title}\par\inserttitle}
\end{textblock}
\begin{textblock}{7.5}(1,7.3)
{\color{white}\raggedright{\insertauthor}\\[0.2cm]
\insertdate}
\end{textblock}}

\setlength\abovedisplayskip{0pt}

\AtBeginSection[] {
  \begin{frame}
  \frametitle{Outline}
    \tableofcontents[currentsection]
  \end{frame}
}

\usetikzlibrary{shapes,arrows}
\tikzstyle{decision} = [diamond, draw, fill=blue!20,
    text width=4.5em, text badly centered, node distance=4cm, inner sep=0pt]
\tikzstyle{block} = [rectangle, draw, fill=blue!20,
    text width=5cm, text centered, rounded corners, minimum height=4em]
\tikzstyle{line} = [draw, thick, -latex']
%\tikzstyle{line} = [->,thi
\tikzstyle{cloud} = [draw, ellipse,fill=red!20, node distance=3cm,
    minimum height=2em, text centered]
\tikzstyle{connector} = [->,thick]

\def\E{\text{E}}
\def\V{\text{Var}}
\def\up#1{\raisebox{-0.3cm}{#1}}

\setlength{\emergencystretch}{0em}
\setlength{\parskip}{0pt}
\def\fullwidth#1{\vspace*{-0.1cm}\par\centerline{\includegraphics[width=12.8cm]{#1}}}
\def\fullheight#1{\vspace*{-0.1cm}\par\centerline{\includegraphics[height=8.5cm]{#1}}}

\fontsize{13}{15}\sf
\usepackage[scale=0.85]{sourcecodepro}
\DisableLigatures{encoding = T1, family = tt*}

% Buchwitz addition
% https://texwelt.de/fragen/2639/wie-kann-ich-kastchenpapier-zeichnen

\newcommand\kariert[2][0.4cm]{%
   \begin{tikzpicture}[gray,step=#1]
     \pgfmathtruncatemacro\anzahl{(\linewidth-\pgflinewidth)/#1} % maximale Anzahl Kästchen pro Zeile
     \draw (0,0) rectangle (\anzahl*#1,#2*#1) (0,0) grid (\anzahl*#1,#2*#1);
   \end{tikzpicture}
}

\newcommand{\liniert}[2][0.5cm]{%
  \begin{tikzpicture}[gray]
  \path[use as bounding box](0,0)rectangle(\linewidth,-#2*#1-0.5\pgflinewidth);
   \foreach \n in {1,...,#2}\draw(0 ,-#1*\n )--(\linewidth,-#1*\n );
  \end{tikzpicture}}
